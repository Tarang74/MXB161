\documentclass{article}
\usepackage{template}

\usepackage[T1]{fontenc}

\usepackage{multicol}
\usepackage{textcomp, upquote}
\usepackage{changepage} % Modify page width

\geometry{
	a4paper,
	margin = 10mm
}

\pagestyle{fancy}
\setlength\headheight{24pt}
\renewcommand{\headrulewidth}{0pt}

\begin{document}
\lstset{language=Matlab, upquote=true}
\lstset{morekeywords={randi, false, imshow, drawpolygon, polyarea, drawline, ones, imread, VideoWriter, XData, YData, getFrame, writeVideo, deg2rad}}

\section*{Random Numbers}
Let $M$, $N$ and $P$ define a $M\times N\times P$ array.
\begin{lstlisting}
rand([M, N, P], 'datatype')    % uniformly distributed random numbers between 0 & 1
randn([M, N, P], 'datatype')   % normally distributed random numbers
\end{lstlisting}
Random numbers between a \& b
\begin{lstlisting}
rand([M, N, P]) * (b - a) + a  % uniformly distributed random numbers
randn([M, N, P]) * (b - a) + a % normally distributed random numbers
randi([a, b], [M, N, P])       % uniformly distributed random integers
\end{lstlisting}
\section*{Data Types}
\begin{table}[H]
    \centering
    \begin{tabular}{c | c | c}
        \toprule
        \textbf{Name} & \textbf{Description} & \textbf{Range} \\
        \midrule
        \lstinline!logical! & boolean values                   & \lstinline!0 & 1! \\
        \lstinline!uint8!   & unsigned 8-bit integers          & \lstinline!0 ... 2^8! \\
        \lstinline!int8!    & unsigned 8-bit integers          & \lstinline!-2^8 ... 2^8! \\
        \lstinline!single!  & single precision ``real" numbers & \lstinline!-realmax ... realmax! \\
        \lstinline!double!  & double precision ``real" numbers & \lstinline!-realmax ... realmax! \\
        \bottomrule
    \end{tabular}
\end{table}
(un)signed 16, 32, 64-bit storage for integer data is created by appending the size to ``(u)int''.
\section*{Operators and Special Characters}
\subsection*{Arithmetic Operators}
MATLAB uses standard mathematical symbols: \lstinline!+!, \lstinline!-!, \lstinline!*!, \lstinline!/!, \lstinline!^!.

\noindent For element-wise operations, prepend the mathematical operator with a dot (\lstinline!.!).
\begin{multicols}{2}
    \subsection*{Relational Operators}
    \begin{table}[H]
        \centering
        \begin{tabular}{c | c}
            \toprule
            \textbf{Symbol} & \textbf{Role} \\
            \midrule
            \lstinline!==! & Equal to \\
            \lstinline!~=! & Not equal to \\
            \lstinline!>!  & Greater than \\
            \lstinline!>=! & Greater than or equal to \\
            \lstinline!<!  & Less than \\
            \lstinline!<=! & Less than or equal to \\
            \bottomrule
        \end{tabular}
    \end{table}
    \columnbreak
    \subsection*{Logical Operators}
    \begin{table}[H]
        \centering
        \begin{tabular}{c | c}
            \toprule
            \textbf{Symbol} & \textbf{Role} \\
            \midrule
            \lstinline!&! & logical \lstinline!AND! \\
            \lstinline!|! & logical \lstinline!OR! \\
            \lstinline!~! & logical \lstinline!NOT! \\
            \bottomrule
        \end{tabular}
    \end{table}
\end{multicols}
\subsection*{Special Characters}
\begin{table}[H]
    \centering
    \begin{tabular}{c | c}
        \toprule
        \textbf{Symbol} & \textbf{Role} \\
        \midrule
        \lstinline!,!   & Separator for row elements \\
        \lstinline!:!   & Index all subscripts in array dimension; create unit-spaced vector \\
        \lstinline!;!   & Separator for column elements; suppress output \\
        \lstinline!( )! & Operator precedence \\
        \lstinline![ ]! & Array creation, multiple output argument assignment \\
        \lstinline!%!   & Comment \\
        \lstinline!""!  & String constructor \\
        \lstinline!~!   & Argument placeholder (suppress specific output) \\
        \lstinline!=!   & Assignment \\
        \bottomrule
    \end{tabular}
\end{table}
\section*{Special Arrays}
\begin{lstlisting}
zeros(M, N) % zero array
false(M, N) % logical false array
\end{lstlisting}
\section*{Array Comparisons}
\begin{lstlisting}
A = rand(M, N); % random array
mask = A > 0.5; % logical array, true if: >0.5 and false if: <=0.5
\end{lstlisting}
\section*{Other Functions}
\begin{lstlisting}
who                     % list workspace variables  
who -file <mat file>    % list variables in .mat file 
pause(x)                % pause procedure for x seconds 
\end{lstlisting}
\section*{Image Processing}
\subsection*{Finding Area}
\begin{lstlisting}
f = figure;                                          % create a figure object
imshow('file.png');                                  % display image 
p = drawpolygon(f.Children);                         % trace polygon on image
cP = p.Position;                                     % n by 2 array of (x, y) coordinates
areaPxSquared = polyarea(cP(:, 1), cP(:, 2));        % area [px^2]
l = drawline(f.Children);                            % trace scale bar on image
cL = l.Position;                                     % 2 by 2 array of (x, y) coordinates
scalePx = sqrt((cL(2, 1) - cL(1, 1))^2 + ...
               (cL(2, 2) - cL(1, 2))^2);             % scale length [px]
mPerPx = actualScaleLength / scalePx;                % [m] per [px]
mSquaredPerPxSquared = mPerPx^2;                     % [m^2] per [px^2]
areaMSquared = mSquaredPerPxSquared * areaPxSquared; % area [m^2]
\end{lstlisting}
\subsection*{Geolocation}
\begin{lstlisting}
longitudes = [...];                                         % e.g. 153.02
latitudes = [...];                                          % e.g. -27.46
origin = [mean(longitudes), mean(latitudes)];               % arbitrary origin
radius = 6373.6;                                            % radius of Earth
circumference = 2 * pi * radius;                            % circumference of Earth
kmPerDegLatitude = circumference / 360;
kmPerDegLongitude = kmPerDegLatitude * cos(deg2rad(-27.5)); % near Brisbane
x = (longitudes - origin(1)) * kmPerDegLongitude;           % x coordinates
y = (latitudes - origin(2)) * kmPerDegLatitude;             % y coordinates
\end{lstlisting}
\section*{Images from Arrays}
\begin{lstlisting}
imshow(A)   % Display image
image(A)    % Display image, recommended if combining with other plots
\end{lstlisting}
\subsection*{Random Images}
\begin{lstlisting}
randi([0, 255], M, N, 'uint8');     % greyscale image
randi([0, 255], M, N, 3, 'uint8');  % colour image
\end{lstlisting}
\subsection*{Creating Colour Images by Modifying Array Entries}
\begin{lstlisting}
A = 255 * zeros(M, N, 3, 'uint8');  % black image
A = 255 * ones(M, N, 3, 'uint8');   % white image
% Access individual channels
rMask = A(:, :, 1);     % red channel
gMask = A(:, :, 2);     % green channel
bMask = A(:, :, 3);     % blue channel
% Access specific region and change its colour to rgb(r, g, b)
A(a:b, c:d, 1) = r;     % modify red value of (a:b, c:d)
A(a:b, c:d, 2) = g;     % modify green value of (a:b, c:d)
A(a:b, c:d, 3) = b;     % modify blue value of (a:b, c:d)
\end{lstlisting}
\subsection*{Editing an Image}
\begin{lstlisting}
theImage = imread('image.png');     % access image 
% Mask a colour range to be modified
mask = theImage(:, :, 1) > r & theImage(:, :, 2) > g & theImage(:, :, 3) > b;
\end{lstlisting}
\subsection*{Create and Save an Animation}
\begin{lstlisting}
f = figure;
set(f, 'Visible', 'on');
video = VideoWriter('video.avi');   % create video object; write to video.avi
x = [...];                          % x values
y = [...];                          % y values
p = plot(x(1), y(1));               % create plot object
for i = 1:length(x)
    % Update plot object data
    p.XData = x(i);
    p.YData = y(i);
    hold on;                        % use if previous points should remain on figure
    drawnow;                        % update figure
    frame = getFrame;               % get snapshot of current axes
    writeVideo(video, frame)        % write frame to video
end
hold off                            % use if hold on was used
close(video);                       % close the file
\end{lstlisting}    
\section*{Sound Processing}
\begin{lstlisting}
f = 523.251
\end{lstlisting}
\end{document}